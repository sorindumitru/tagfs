\newcommand{\fig}[4][]{
\begin{figure}[htb]
	\begin{center}
		\includegraphics[#1]{#2}
		\caption{#4 \label{#3}}
    \end{center}
\end{figure}
}

% bare_conf.tex
%% V1.3
%% 2007/01/11
%% by Michael Shell
%% See:
%% http://www.michaelshell.org/
%% for current contact information.
%%
%% This is a skeleton file demonstrating the use of IEEEtran.cls
%% (requires IEEEtran.cls version 1.7 or later) with an IEEE conference paper.
%%
%% Support sites:
%% http://www.michaelshell.org/tex/ieeetran/
%% http://www.ctan.org/tex-archive/macros/latex/contrib/IEEEtran/
%% and
%% http://www.ieee.org/

%%*************************************************************************
%% Legal Notice:
%% This code is offered as-is without any warranty either expressed or
%% implied; without even the implied warranty of MERCHANTABILITY or
%% FITNESS FOR A PARTICULAR PURPOSE! 
%% User assumes all risk.
%% In no event shall IEEE or any contributor to this code be liable for
%% any damages or losses, including, but not limited to, incidental,
%% consequential, or any other damages, resulting from the use or misuse
%% of any information contained here.
%%
%% All comments are the opinions of their respective authors and are not
%% necessarily endorsed by the IEEE.
%%
%% This work is distributed under the LaTeX Project Public License (LPPL)
%% ( http://www.latex-project.org/ ) version 1.3, and may be freely used,
%% distributed and modified. A copy of the LPPL, version 1.3, is included
%% in the base LaTeX documentation of all distributions of LaTeX released
%% 2003/12/01 or later.
%% Retain all contribution notices and credits.
%% ** Modified files should be clearly indicated as such, including  **
%% ** renaming them and changing author support contact information. **
%%
%% File list of work: IEEEtran.cls, IEEEtran_HOWTO.pdf, bare_adv.tex,
%%                    bare_conf.tex, bare_jrnl.tex, bare_jrnl_compsoc.tex
%%*************************************************************************

% *** Authors should verify (and, if needed, correct) their LaTeX system  ***
% *** with the testflow diagnostic prior to trusting their LaTeX platform ***
% *** with production work. IEEE's font choices can trigger bugs that do  ***
% *** not appear when using other class files.                            ***
% The testflow support page is at:
% http://www.michaelshell.org/tex/testflow/



% Note that the a4paper option is mainly intended so that authors in
% countries using A4 can easily print to A4 and see how their papers will
% look in print - the typesetting of the document will not typically be
% affected with changes in paper size (but the bottom and side margins will).
% Use the testflow package mentioned above to verify correct handling of
% both paper sizes by the user's LaTeX system.
%
% Also note that the "draftcls" or "draftclsnofoot", not "draft", option
% should be used if it is desired that the figures are to be displayed in
% draft mode.
%
\documentclass[conference]{IEEEtran}
% Add the compsoc option for Computer Society conferences.
%
% If IEEEtran.cls has not been installed into the LaTeX system files,
% manually specify the path to it like:
% \documentclass[conference]{../sty/IEEEtran}





% Some very useful LaTeX packages include:
% (uncomment the ones you want to load)


% *** MISC UTILITY PACKAGES ***
%
%\usepackage{ifpdf}
% Heiko Oberdiek's ifpdf.sty is very useful if you need conditional
% compilation based on whether the output is pdf or dvi.
% usage:
% \ifpdf
%   % pdf code
% \else
%   % dvi code
% \fi
% The latest version of ifpdf.sty can be obtained from:
% http://www.ctan.org/tex-archive/macros/latex/contrib/oberdiek/
% Also, note that IEEEtran.cls V1.7 and later provides a builtin
% \ifCLASSINFOpdf conditional that works the same way.
% When switching from latex to pdflatex and vice-versa, the compiler may
% have to be run twice to clear warning/error messages.






% *** CITATION PACKAGES ***
%
%\usepackage{cite}
% cite.sty was written by Donald Arseneau
% V1.6 and later of IEEEtran pre-defines the format of the cite.sty package
% \cite{} output to follow that of IEEE. Loading the cite package will
% result in citation numbers being automatically sorted and properly
% "compressed/ranged". e.g., [1], [9], [2], [7], [5], [6] without using
% cite.sty will become [1], [2], [5]--[7], [9] using cite.sty. cite.sty's
% \cite will automatically add leading space, if needed. Use cite.sty's
% noadjust option (cite.sty V3.8 and later) if you want to turn this off.
% cite.sty is already installed on most LaTeX systems. Be sure and use
% version 4.0 (2003-05-27) and later if using hyperref.sty. cite.sty does
% not currently provide for hyperlinked citations.
% The latest version can be obtained at:
% http://www.ctan.org/tex-archive/macros/latex/contrib/cite/
% The documentation is contained in the cite.sty file itself.






% *** GRAPHICS RELATED PACKAGES ***
%
%\usepackage[absolute,overlay]{textpos}
\usepackage[pdftex]{graphicx}
\usepackage{url}
\ifCLASSINFOpdf
  % \usepackage[pdftex]{graphicx}
  % declare the path(s) where your graphic files are
  % \graphicspath{{../pdf/}{../jpeg/}}
  % and their extensions so you won't have to specify these with
  % every instance of \includegraphics
  % \DeclareGraphicsExtensions{.pdf,.jpeg,.png}
\else
  % or other class option (dvipsone, dvipdf, if not using dvips). graphicx
  % will default to the driver specified in the system graphics.cfg if no
  % driver is specified.
  % \usepackage[dvips]{graphicx}
  % declare the path(s) where your graphic files are
  % \graphicspath{{../eps/}}
  % and their extensions so you won't have to specify these with
  % every instance of \includegraphics
  % \DeclareGraphicsExtensions{.eps}
\fi
% graphicx was written by David Carlisle and Sebastian Rahtz. It is
% required if you want graphics, photos, etc. graphicx.sty is already
% installed on most LaTeX systems. The latest version and documentation can
% be obtained at: 
% http://www.ctan.org/tex-archive/macros/latex/required/graphics/
% Another good source of documentation is "Using Imported Graphics in
% LaTeX2e" by Keith Reckdahl which can be found as epslatex.ps or
% epslatex.pdf at: http://www.ctan.org/tex-archive/info/
%
% latex, and pdflatex in dvi mode, support graphics in encapsulated
% postscript (.eps) format. pdflatex in pdf mode supports graphics
% in .pdf, .jpeg, .png and .mps (metapost) formats. Users should ensure
% that all non-photo figures use a vector format (.eps, .pdf, .mps) and
% not a bitmapped formats (.jpeg, .png). IEEE frowns on bitmapped formats
% which can result in "jaggedy"/blurry rendering of lines and letters as
% well as large increases in file sizes.
%
% You can find documentation about the pdfTeX application at:
% http://www.tug.org/applications/pdftex





% *** MATH PACKAGES ***
%
%\usepackage[cmex10]{amsmath}
% A popular package from the American Mathematical Society that provides
% many useful and powerful commands for dealing with mathematics. If using
% it, be sure to load this package with the cmex10 option to ensure that
% only type 1 fonts will utilized at all point sizes. Without this option,
% it is possible that some math symbols, particularly those within
% footnotes, will be rendered in bitmap form which will result in a
% document that can not be IEEE Xplore compliant!
%
% Also, note that the amsmath package sets \interdisplaylinepenalty to 10000
% thus preventing page breaks from occurring within multiline equations. Use:
%\interdisplaylinepenalty=2500
% after loading amsmath to restore such page breaks as IEEEtran.cls normally
% does. amsmath.sty is already installed on most LaTeX systems. The latest
% version and documentation can be obtained at:
% http://www.ctan.org/tex-archive/macros/latex/required/amslatex/math/





% *** SPECIALIZED LIST PACKAGES ***
%
%\usepackage{algorithmic}
% algorithmic.sty was written by Peter Williams and Rogerio Brito.
% This package provides an algorithmic environment fo describing algorithms.
% You can use the algorithmic environment in-text or within a figure
% environment to provide for a floating algorithm. Do NOT use the algorithm
% floating environment provided by algorithm.sty (by the same authors) or
% algorithm2e.sty (by Christophe Fiorio) as IEEE does not use dedicated
% algorithm float types and packages that provide these will not provide
% correct IEEE style captions. The latest version and documentation of
% algorithmic.sty can be obtained at:
% http://www.ctan.org/tex-archive/macros/latex/contrib/algorithms/
% There is also a support site at:
% http://algorithms.berlios.de/index.html
% Also of interest may be the (relatively newer and more customizable)
% algorithmicx.sty package by Szasz Janos:
% http://www.ctan.org/tex-archive/macros/latex/contrib/algorithmicx/




% *** ALIGNMENT PACKAGES ***
%
%\usepackage{array}
% Frank Mittelbach's and David Carlisle's array.sty patches and improves
% the standard LaTeX2e array and tabular environments to provide better
% appearance and additional user controls. As the default LaTeX2e table
% generation code is lacking to the point of almost being broken with
% respect to the quality of the end results, all users are strongly
% advised to use an enhanced (at the very least that provided by array.sty)
% set of table tools. array.sty is already installed on most systems. The
% latest version and documentation can be obtained at:
% http://www.ctan.org/tex-archive/macros/latex/required/tools/


%\usepackage{mdwmath}
%\usepackage{mdwtab}
% Also highly recommended is Mark Wooding's extremely powerful MDW tools,
% especially mdwmath.sty and mdwtab.sty which are used to format equations
% and tables, respectively. The MDWtools set is already installed on most
% LaTeX systems. The lastest version and documentation is available at:
% http://www.ctan.org/tex-archive/macros/latex/contrib/mdwtools/


% IEEEtran contains the IEEEeqnarray family of commands that can be used to
% generate multiline equations as well as matrices, tables, etc., of high
% quality.


%\usepackage{eqparbox}
% Also of notable interest is Scott Pakin's eqparbox package for creating
% (automatically sized) equal width boxes - aka "natural width parboxes".
% Available at:
% http://www.ctan.org/tex-archive/macros/latex/contrib/eqparbox/





% *** SUBFIGURE PACKAGES ***
%\usepackage[tight,footnotesize]{subfigure}
% subfigure.sty was written by Steven Douglas Cochran. This package makes it
% easy to put subfigures in your figures. e.g., "Figure 1a and 1b". For IEEE
% work, it is a good idea to load it with the tight package option to reduce
% the amount of white space around the subfigures. subfigure.sty is already
% installed on most LaTeX systems. The latest version and documentation can
% be obtained at:
% http://www.ctan.org/tex-archive/obsolete/macros/latex/contrib/subfigure/
% subfigure.sty has been superceeded by subfig.sty.



%\usepackage[caption=false]{caption}
%\usepackage[font=footnotesize]{subfig}
% subfig.sty, also written by Steven Douglas Cochran, is the modern
% replacement for subfigure.sty. However, subfig.sty requires and
% automatically loads Axel Sommerfeldt's caption.sty which will override
% IEEEtran.cls handling of captions and this will result in nonIEEE style
% figure/table captions. To prevent this problem, be sure and preload
% caption.sty with its "caption=false" package option. This is will preserve
% IEEEtran.cls handing of captions. Version 1.3 (2005/06/28) and later 
% (recommended due to many improvements over 1.2) of subfig.sty supports
% the caption=false option directly:
%\usepackage[caption=false,font=footnotesize]{subfig}
%
% The latest version and documentation can be obtained at:
% http://www.ctan.org/tex-archive/macros/latex/contrib/subfig/
% The latest version and documentation of caption.sty can be obtained at:
% http://www.ctan.org/tex-archive/macros/latex/contrib/caption/




% *** FLOAT PACKAGES ***
%
%\usepackage{fixltx2e}
% fixltx2e, the successor to the earlier fix2col.sty, was written by
% Frank Mittelbach and David Carlisle. This package corrects a few problems
% in the LaTeX2e kernel, the most notable of which is that in current
% LaTeX2e releases, the ordering of single and double column floats is not
% guaranteed to be preserved. Thus, an unpatched LaTeX2e can allow a
% single column figure to be placed prior to an earlier double column
% figure. The latest version and documentation can be found at:
% http://www.ctan.org/tex-archive/macros/latex/base/



%\usepackage{stfloats}
% stfloats.sty was written by Sigitas Tolusis. This package gives LaTeX2e
% the ability to do double column floats at the bottom of the page as well
% as the top. (e.g., "\begin{figure*}[!b]" is not normally possible in
% LaTeX2e). It also provides a command:
%\fnbelowfloat
% to enable the placement of footnotes below bottom floats (the standard
% LaTeX2e kernel puts them above bottom floats). This is an invasive package
% which rewrites many portions of the LaTeX2e float routines. It may not work
% with other packages that modify the LaTeX2e float routines. The latest
% version and documentation can be obtained at:
% http://www.ctan.org/tex-archive/macros/latex/contrib/sttools/
% Documentation is contained in the stfloats.sty comments as well as in the
% presfull.pdf file. Do not use the stfloats baselinefloat ability as IEEE
% does not allow \baselineskip to stretch. Authors submitting work to the
% IEEE should note that IEEE rarely uses double column equations and
% that authors should try to avoid such use. Do not be tempted to use the
% cuted.sty or midfloat.sty packages (also by Sigitas Tolusis) as IEEE does
% not format its papers in such ways.





% *** PDF, URL AND HYPERLINK PACKAGES ***
%
%\usepackage{url}
% url.sty was written by Donald Arseneau. It provides better support for
% handling and breaking URLs. url.sty is already installed on most LaTeX
% systems. The latest version can be obtained at:
% http://www.ctan.org/tex-archive/macros/latex/contrib/misc/
% Read the url.sty source comments for usage information. Basically,
% \url{my_url_here}.





% *** Do not adjust lengths that control margins, column widths, etc. ***
% *** Do not use packages that alter fonts (such as pslatex).         ***
% There should be no need to do such things with IEEEtran.cls V1.6 and later.
% (Unless specifically asked to do so by the journal or conference you plan
% to submit to, of course. )


% correct bad hyphenation here
\hyphenation{op-tical net-works semi-conduc-tor}


\begin{document}
%
% paper title
% can use linebreaks \\ within to get better formatting as desired
\title{TagFS - A Tag Based Filesystem}


% author names and affiliations
% use a multiple column layout for up to three different
% affiliations
\author{\IEEEauthorblockN{Catalina Macalet}
\IEEEauthorblockA{Computer Science Department\\
Politehnica University of Bucharest\\
Email: catalina.macalet@cti.pub.ro}
\and
\IEEEauthorblockN{Eugen Hristev}
\IEEEauthorblockA{Computer Science Department\\
Politehnica University of Bucharest\\
Email: eugen.hristev@cti.pub.ro}
\and
\IEEEauthorblockN{Mihai Dinu}
\IEEEauthorblockA{Computer Science Department\\
Politehnica University of Bucharest\\
Email: mihai.dinu@cti.pub.ro}
\and
\IEEEauthorblockN{Sorin Dumitru}
\IEEEauthorblockA{Computer Science Department\\
Politehnica University of Bucharest\\
Email: sorin.dumitru@cti.pub.ro}}

% conference papers do not typically use \thanks and this command
% is locked out in conference mode. If really needed, such as for
% the acknowledgment of grants, issue a \IEEEoverridecommandlockouts
% after \documentclass

% for over three affiliations, or if they all won't fit within the width
% of the page, use this alternative format:
% 
%\author{\IEEEauthorblockN{Michael Shell\IEEEauthorrefmark{1},
%Homer Simpson\IEEEauthorrefmark{2},
%James Kirk\IEEEauthorrefmark{3}, 
%Montgomery Scott\IEEEauthorrefmark{3} and
%Eldon Tyrell\IEEEauthorrefmark{4}}
%\IEEEauthorblockA{\IEEEauthorrefmark{1}School of Electrical and Computer Engineering\\
%Georgia Institute of Technology,
%Atlanta, Georgia 30332--0250\\ Email: see http://www.michaelshell.org/contact.html}
%\IEEEauthorblockA{\IEEEauthorrefmark{2}Twentieth Century Fox, Springfield, USA\\
%Email: homer@thesimpsons.com}
%\IEEEauthorblockA{\IEEEauthorrefmark{3}Starfleet Academy, San Francisco, California 96678-2391\\
%Telephone: (800) 555--1212, Fax: (888) 555--1212}
%\IEEEauthorblockA{\IEEEauthorrefmark{4}Tyrell Inc., 123 Replicant Street, Los Angeles, California 90210--4321}}




% use for special paper notices
%\IEEEspecialpapernotice{(Invited Paper)}




% make the title area
\maketitle


\begin{abstract}
%\boldmath

File systems are an integral part of every operating system. Because of the 
high capacity of modern storing devices file systems need a better way of 
organizing and accessing data in order to be easier for one to retrieve
exactly the files he/she is looking for.
Also, the need of users to personalize the content stored and
to find specific data, pushes manufacturers to employ alternatives for the 
curent design.

TagFS implements a tag based file system in Linux which offers support for 
tagging files and browsing files by tags.

\end{abstract}
% keywords

\begin{keywords}
File systems, Tags, VFS, metadata
\end{keywords}    

% For peer review papers, you can put extra information on the cover
% page as needed:
% \ifCLASSOPTIONpeerreview
% \begin{center} \bfseries EDICS Category: 3-BBND \end{center}
% \fi
%
% For peerreview papers, this IEEEtran command inserts a page break and
% creates the second title. It will be ignored for other modes.
\IEEEpeerreviewmaketitle



\section{Introduction}
In most operating systems the files are hierarchically organized. 
This means that there usually is a starting point, or parent directory. 
In Windows based systems there are multiple starting points 
based on the physical hard drive partitions. In Unix-like systems, 
there is a single root drive with different mount points available for users
to add or remove subtrees from different drives, partitions, etc. 
In these filesystems a user organizes related data by storing it in the same 
folder but say that a user, Bob, has two separated folders one for storing
photos taken in the mountains (Mountain-pics) and one for storing photos in which 
a certain person appears (Alice-pics). Two questions arise, one, where should 
Bob store a picture taken in the mountains in which Alice appears and two, how 
could Bob find the pictures taken in the mountains in which Alice appears. For 
current
filesystems the answer to the first question might be storing the photo in either
folder and in the second one creating a link to this photo or, store it in both
folders. The answer to the second one could be naming the photo in such a way
that retrieving them based on the previously stated criteria would work. 
TagFS file system aims to bring a different approach, based on tags rather than 
hierarchical system that is rooted for a long time in modern operating system. 
For the above example, for a TagFS filesystem the answer to both questions would
be adding tags to photos ($<$mountains$><$Alice$>$) and then search for files that 
contain these tags. The question of where to store a specific photo 
 would not be that important anymore. 
A pure tag file system is difficult to implement starting from zero, so we tried
to adapt the current file system in Linux to support tags and see how the two 
systems can coexist on an end-user machine. 
A tag file system should be able to organize files, data on the disk regardless of 
hierarchical logical approach. The position of the files on the disk is irrelevant and 
completely transparent to the user. The file system should be able to put files on disks
and simply recover them on demand based on tags requests. 
In our approach, logical directory based organization and file tags coexist, in order
to see how the two systems can fit and how the user can use alternatives for searching
and clustering the information it has. 
We implemented a tag layer in the Linux Virtual File System and tested how this impacts 
the regular user. We've added posibilities for the user to manipulate the tags (add, delete,
 search) in order to increase the flexibility of the filesystem and the way it interacts with the user.
TagFS is expected to make it easier to work with files, especially personal ones.

%\subsection{Subsection Heading Here}
%Subsection text here.


%\subsubsection{Subsubsection Heading Here}
%Subsubsection text here.

\section{State of the art}

The idea of tagging files in order to access them in an easier fashion is not
a new one and various attempts to implement solutions have been made. Some of
these are specialized solutions for special kind of data, such as Calibre
which makes ebook management easier, implemented in userspace. The vast majority
of these applications rely on a database where mappings between files and 
associated metadata are stored and expose a set of commands which translate
to specific queries for the database.  

\subsection[Nepomuk-KDE]{Nepomuk-KDE\footnote{\url{http://nepomuk.kde.org/}}}
Nepomuk-KDE is an implementation of Nepomuk which has been integrated with KDE
and that allows adding metadata to items stored on a computer and making 
queries based on that metadata.
Dolphin KDE file manager allows adding and removing basic metadata to files,
such as comments and tags.
\todo{more on this}

\subsection{Other Projects}
\textit{TagFS: Bringing Semantic Metadata to the Filesystem}\footnote{ \url{http:
//www.eswc2006.org/poster-papers/FP31-Schenk.pdf/}} is a research projected
started at the University of Koblenz which, as Nepomuk, relies on RDF for
defining semantics and SPARQL. Metadata is stored in a repository having an
associated graph, and various opperations can be performed on it(additions, 
updates, etc).
$\\$
\textit{TaggedFrog}\footnote{ \url{http://lunarfrog.com/}} is a Windows application 
based on the convenint drag'n'drop technique. It allows you to organize your files, 
documents and Web links just by adding objects to the library and tagging them with 
any keywords. Moreover, you are able tagging files directly from Windows File Explorer 
because the application is integrated with Explorer's context menu.
$\\$
\textit{pytagsfs}\footnote{ \url{http://www.pytagsfs.org/}}
is a FUSE filesystem, written in python for Linux and Mac OS X systems, that 
arranges media files in a virtual directory structure based on the file tags. 
File tags can be changed by moving and renaming virtual files and directories. 
The virtual files can also be modified directly, and, of course, can be opened 
and played just like regular files.

\section{Tagfs}
    
TagFS is a software application that implements a tag-based filesystem in 
Linux, more specifically, TagFS allows tagging a file at creation time or at
a later time, adding and removing tags, listing the tags associated to a file
at a given time and, the most important characteristic, TagFS allows
browsing for files having specific tag(s).

The filesystem hierarchy whill remain unchanged but files will have
associated tags (an example is presented in \figref{img:hierarchy}).
\fig[scale=0.5]{figs/hierarchy.pdf}{img:hierarchy}{TagFS hierarchy}

The TagFS application architecture is presented in \figref{arch}. 
\fig[scale=0.5]{figs/archall.pdf}{arch}{TagFS architecture}
The \textit{CLI} is used to issue commands for tag manipulation. There are two types
of commands. The first type consists of file manipulation commands available on 
every Unix-like operating system, such as \textit{ls, touch, mv, cp} whose behaviour 
and implementation was changed in TagFS implementation in order to support tags. 
The second type of commands reffers to new TagFS commands implemented in order to 
provide more tag-related operations.  
The implementation changes are related to hooks created in \textit{VFS} and will be detailed in 
subsection \textit{Tag handling}. 
No implementation changes at filesystem level were required.

\subsection{Architectural decisions}
TagFS started as an idea to create a more user-friendly file system; remembering
tags is easier than remembering the name of a file or the place where it is stored
but, at the same time a pure tag filesystem would offer no simple way of organizing data 
in a hierarchical manner, a choice to implement TagFS as a new filesystem, from 
scratch, thus would have been pointless.
The other choice was to implement TagFS as hooks in
VFS in order to store and retrieve needed metadata.
Since changes are made at VFS level there will be an overhead for filesystems that
subsequently are to be used without tag support. A main concern in implementation
was to reduce this overhead to as little as possible.
$\\$
From the beginning the focus was on the changes needed at VFS and file system 
level and not on storage possibilities of the mappings between files and
associated tags and so these mappings are stored in a file which is always in RAM memory.
$\\$
The keyword of entry point was defined in order to designate a point in the file system
hierarchy starting from which a distinct TagFS begins meaning that only for that part of
the file system tags apply; for a file system multiple entry points can be declared.
\todo{Sorin is this assumption correct?}
      
\subsection{Storage}
The mapping between a file and its associated tags is presented in \figref{storage}.
\fig[scale=0.4]{figs/storage.pdf}{storage}{Tag Storage}
Actually it maps tags to files and this is because not necessary all files from a TagFS
have tags associated.
Since a file can be tagged more than once, for every tag a list of pointers to
files is stored rather than actual files.
\todo{Reformulati va rog}
The implementation details are presented in a separate section later in this
paper.

\subsection{Tag commands}
The idea of tagging files is to be able to add a number of tags to a file but
since the number of tags that will be associated to a file is not known beforehand
we establish a convention that is that a filename will be separated from its 
associated tags by ":" whenever a command that envolves tags is issued. Also,
one tag is separated by another tag by ":".
$\\$
The behaviour of \textit{ls} command so that when issued whith an argument
starting with ":" it lists all the files that are tagged with the given words.
$\\$
An existent file can be assigned tags by issueing the \textit{tag} command
with \textit{-a} parameter followed by the filename and the tags that one wants to
attach to that file. There are two constraints that one has to take into
account when wanting to add tags to a file. One is that the implementation of
TagFS limits the maximum number of tags that can be assigned to a file to 256
and the second one is a limitation imposed by the kernel implementation and
it reffers to the fact that the total length of filename, tags and separator
must be less or equal to the value of MAX$\_$PATH$\_$LENGTH(256).
$\\$
Tags can be removed from a file with \textit{tag -d filename:tag[:tag*]} command
taking into consideration the second constraint stated above.
$\\$
The output of \textit{tag -l filename} command is the list of tags associated
to a file at a given time.
$\\$
TagFS permits the creation of entry points in the file system which indicate 
that starting from that point down the hierarchy tags may be used.
This was introduced in order to reduce the overhead for file systems where
the user does not require to use tags, limiting it to a couple of comparisons.
An entry point can be created using \textit{tag -c} command meaning that
the current working directory is a new TagFS entry point.  
$\\$
The available commands as well as a short description is listed in Table1.

\begin{center}
  \begin{table}[htb]
  \begin{center}
  \begin{tabular}{ | l | l | l | l |}
    \hline
      \textbf{Command}&\textbf{Params} &\textbf{Args}&\textbf{Description}\\ \hline
        ls  & -  & :tag[:tag]*         & List files having the specified tags\\ \hline
        tag & -c & -                   & Create a new entry point for TagFS\\ \hline
        tag & -a & filename:tag[:tag]* & Add tags to file\\ \hline
        tag & -d & filename:tag[:tag]* & Remove tags from filename\\ \hline
        tag & -l & filename            & List all the tags for filename\\
    \hline
  \end{tabular}
  \end{center}
  \caption{TagFS implemented commands}
  \label{table:commands}
  \end{table}
\end{center}

\subsection{Implementation details}
\subsubsection[medatada]{Metadata structure}$\\$
At the moment of allocation in the system each tag is associated an unique number
\todo{metadata implementation}
\subsubsection{VFS Hooks} $\\$
The VFS hooks allow TagFS to break out of the normal flow of the kernel and performs certain
verifications in order to determine if a particular file should or not be treated as a tag-able
file and afterwards, if necessary, performing the desired changes.
\todo{hooks}
\subsubsection{Userspace application} 
$\\$The application in userspace adds the tag layer to the normal file operations using the tag
command. This command is a normal user space application that can be called by the user. 
This way the user can add and
remove tags from a file, and also list the current tags. The application allows creating an entry point in the file
system in a similar way. Tagging application works in user space and calls the specific api
that further sends the requests to the kernel. The adding and removing of the tags keep the 
specified convention, using ":" as delimitator. The tag listing keeps the same format as well.
The tag user application uses \textit{fcntl} system call in order to send the operation type
and required arguments, new command types have been defined for fcntl for TagFS operations. 
From fcntl syscall other kernel-level tag specific functions are called.
\todo{sorin, fie zici aici de functiile noi fie daca le-ai amintit la VFS hooks zici
ca sunt cele de mai sus}

\section{Conclusion}

$\\$
Possible future work:
In a pure tag file system, the disk mechanism could be improved in the following way:
We know that tags can be added to some files, we have no hierarchical structure of the files.
This way we can find blocks of files based on tags which could reduce disk fragmentation. 
Clustering tag data can give insight on how much space there is required of a certain tag type files
and how accessible this should be to the user. This could lower the external fragmentation of the disk
if properly used. However more tests should be done regarding this problem. 
$\\ \\$
Given the fact that Linux implements Extended Attributes(also called xattrs) which are name/value pairs 
associated with files as an extension to normal inode-based attributes, tags could be inserted in xattrs
and could be easily displayed of graphical file browsers.
\todo{nu e file browsers, e altceva da e prea tarziu}

\begin{thebibliography}{1}
	\bibitem SStephan Bloehdorn and Max Volkel,Tagfs-tag semantics for hierarchical file systems, 
	\bibitem YYuan-Liang Tai1, Shang-Rong Tsai, Guang-Hung Huang, Chia-Ming Lee,
Lian-Jou Tsai, Kuo-Feng Ssu and Shou-Jen Wey, A Label-Based File System
    \bibitem NNepomuk-KDE \url{http://nepomuk.kde.org/}
\end{thebibliography}

% that's all folks
\end{document}


